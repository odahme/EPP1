\documentclass[10pt]{article}

%% Text-Encoding festlegen. Mit utf8 oder utf8x funktionieren Umlaute wie gewohnt.
%% (mit Bibtex funktioniert nur utf8)
\usepackage[utf8x]{inputenc}

%% Sprachdatei für Trennregeln, Datum-Format und ähnliches festlegen
%\usepackage[german]{babel}  % nötig für Umlaute
 \usepackage[english]{babel}

%% optimiert das typographische Erscheinungsbild
\usepackage{microtype}

%% erlaubt Listen einfacher zu formatieren (bietet nosep für kompakte Listen)
\usepackage{enumitem}
%% erlaubt hübsche Tabellen über mehrere Seiten, beinhaltet booktabs (\toprule, \midrule, ...)
\usepackage{ctable}
%% ermöglicht farbigen Text ({\color{red} ...})
\usepackage{xcolor}

%% erweiterte Funktionalität für Formeln (Pakete der American Mathematical Society)
\usepackage{amsfonts,amsmath,amsthm,amssymb}

%% vordefinierte Einheiten, einfaches Angeben von Einheiten (\SI{8 \pm 1}{cm})
%%   die Unsicherheit soll mit +- abgetrennt werden
\usepackage[separate-uncertainty]{siunitx}
\sisetup{
    range-units = single,       % \SIrange soll die Einheit nur einmal anzeigen
    list-units  = repeat,       % \SIlist soll die Einheit wiederholen
}
%% bei siuntix funktioniert babel leider nicht
%% für englische Dokumente sollten diese Zeilen auskommentiert werden.
%\sisetup{
%    range-phrase         = { bis },
%    list-final-separator = { und },
%    list-pair-separator  = { und }, % an Uni noch nicht verfügbar
%}

%% erlaubt es Bilddateien einzubinden
%% (ctable graphicx intern auch. Trotzdem ist es sinnvoll graphicx expilizt zu laden.
%%  Sonst entstehen schwehr verständliche Fehler, wenn ctable entfernt wird)
\usepackage{graphicx}
%% ermöglicht Bilder und Tabellen am eingegebenen Ort zu platzieren ([H])
\usepackage{float}
%% ermöglicht Unter-Bilder in einer figure-Umgebung
\usepackage{subfig}
%% Grafik-Dateien werden in den folgenden Ordnern gesucht
\graphicspath{{img/}}
%% Grafikdateien haben die folgenden Endungen (höchste Priorität zu erst)
\DeclareGraphicsExtensions{.pdf,.png,.jpg}

%% Vertikaler Abstand zwischen Absätzen, Beginn eines Absatzes nicht einrücken
\usepackage{parskip}
% \setlength{\parskip}{0.6em}   % Vertikaler Abstand zwischen Absätzen anpassen
% \setlength{\parindent}{0em}   % Einrück-Abstand anpassen

%% zeige Labels im Seitenrand. Dies ist praktisch um Verweise zu kontrollieren
\usepackage[final]{showkeys} % die Option 'final' deaktiviert die Ausgabe von showkeys

%% Seiten-Layout einstellen
\usepackage[
 a4paper,
 total={16cm,26cm},          % Breite und Höhe des Inhalt-Bereichs
 top=20mm, left=30mm,        % Ränder oben und links
 headsep=10mm,               % Abstand des unteren Rands der Kopfzeile vom oberen Rand des Inhalts
 footskip=10mm               % Abstand des unteren des Inhalts zum oberen Rand der Fusszeile
]{geometry}

%% Ermöglicht Links im PDF
%%   sollte möglichst spät in der Präambel geladen werden
\usepackage[
 pdftex,                        % wir verwenden pdftex/pdflatex
 bookmarks=true,                % wir wollen auch im PDF-Reader ein Inhaltsverzeichnis
 bookmarksdepth=3,              % das Inhaltsverzeichnis soll 3 Tiefen enthalten
 colorlinks=true,               % Linktexte sollen Farbig sein
 linkcolor=black,               % Links innerhalb des Dokuments bleiben schwarz
 citecolor=black,               % Links zu Quellenangaben bleiben ebenfalls schwarz
 urlcolor=blue,                 % URL-Linktexte sollen blau dargestellt werden
%  pdfborder={0 0 0}              % Links im PDF erhalten keinen Rahmen, nur nötig wenn colorlinks=false
]{hyperref}

%% definiert \cref: Referenzen mit korrekter Bezeichnung (z.B. "Abbildung 1")
%%   die Nummer alleine ist weiter mittels \ref verfügbar
%% muss NACH 'hyperref' geladen werden
%\usepackage[german]{cleveref}
 \usepackage[english, capitalise]{cleveref}


%% Angaben für \maketitle
\title{Measuring of the invariant top mass}
\author{Oliver Dahme}
% \date{7. Mai 2013}             % ohne Angabe wird das heutig Datum verwendet

%% Angaben für die PDF-Eigenschaften
\makeatletter
\hypersetup{
  pdfauthor = {\@author},
  pdftitle = {\@title},
}
\makeatother

\begin{document}

\maketitle

\begin{abstract}
The top quark mass was measured on a small dataset of the CMS detector at CERN. Therfore a selection of the data was performed and tested by a Monte Carlo (MC) simulation. First the cross section was measured then the Z-boson mass and the W-boson mass. At the end the top mass was measured out of the hadronic part of the decay and a second measuerement out of the semi-leptonic part of the decay.
\end{abstract}

\tableofcontents

\newpage

\section{Introduction}
To measure the mass of the top quark a selection has been performed to selected events of the form:
\begin{figure}[H]
\centering
%% Die dreifachen Klammern {{{ }}} sind nötig, damit Latex auch mit
%% Dateinamen umgehen kann, die Punkte enthalten.
\includegraphics[width=0.45\textwidth]{{{top_decay}}}
\caption{Hadronic and Semileptonic decay of a top, anti top pair}
\label{fig:top}
\end{figure}
In the following plots one can see the Number of Jets, the Number of b tagged jets and the missing transverse energy distribution bevor the selection and after. \\
An event is accepted if:
\begin{enumerate}
  \item It has 4 Jets in the final state
  \item It has at least one isolated muon in the final state
  \item It has at least two b-tagged jets
\end{enumerate}
This selection leads to the following numbers:
\begin{table}[tph]
\centering
\begin{tabular}{l|r|r}
\toprule
Name & Value & Error\\
\midrule
Total number of MC simulated events & 240601 & 490 \\
Number of MC triggerd events & 208175 & 456  \\
Total number of MC simulated $t \bar{t}$ events & 36941 & 192\\
Number of MC accepted $t \bar{t}$ events & 506 & 23\\
Number of expected $t \bar{t}$-events in Data & 6429 & 80 \\
Number of accepted events in Data & 75 & 9 \\
\bottomrule
\end{tabular}
\caption{Summary of number of events}
\label{tab:numbers}
\end{table}



\begin{figure}[H]
\centering
%% Die dreifachen Klammern {{{ }}} sind nötig, damit Latex auch mit
%% Dateinamen umgehen kann, die Punkte enthalten.
\includegraphics[width=0.75\textwidth]{{{NJet_ini}}}
\caption{Number of jets before selection}
\label{fig:njet_i}
\end{figure}
\begin{figure}[H]
\centering
%% Die dreifachen Klammern {{{ }}} sind nötig, damit Latex auch mit
%% Dateinamen umgehen kann, die Punkte enthalten.
\includegraphics[width=0.75\textwidth]{{{NJet_final}}}
\caption{Number of jets after selection}
\label{fig:njet_f}
\end{figure}
\begin{figure}[H]
\centering
%% Die dreifachen Klammern {{{ }}} sind nötig, damit Latex auch mit
%% Dateinamen umgehen kann, die Punkte enthalten.
\includegraphics[width=0.75\textwidth]{{{NbJet_ini}}}
\caption{Number of b-jets before selection}
\label{fig:nbjet_i}
\end{figure}
\begin{figure}[H]
\centering
%% Die dreifachen Klammern {{{ }}} sind nötig, damit Latex auch mit
%% Dateinamen umgehen kann, die Punkte enthalten.
\includegraphics[width=0.75\textwidth]{{{NbJet_final}}}
\caption{Number of b-jets after selection}
\label{fig:nbjet_f}
\end{figure}
\begin{figure}[H]
\centering
%% Die dreifachen Klammern {{{ }}} sind nötig, damit Latex auch mit
%% Dateinamen umgehen kann, die Punkte enthalten.
\includegraphics[width=0.75\textwidth]{{{met_ini}}}
\caption{Missing transverse energy distribution before selection}
\label{fig:met_i}
\end{figure}

\begin{figure}[H]
\centering
%% Dateinamen umgehen kann, die Punkte enthalten.
\includegraphics[width=0.75\textwidth]{{{met_final}}}
\caption{Missing transverse energy distribution after selection}
\label{fig:met_f}
\end{figure}

\section{Cross section}
In this section the cross section for $\sigma_{t \bar{t}}$ at 7 TeV is calculated:
\begin{align}
  \sigma_{t \bar{t}} = \frac{N}{\epsilon_{trigger} \cdot L \cdot A_{t \bar{t}}}
\end{align}
where $N$ is number of accepted events in the Data, $\epsilon_{trigger}$ is the trigger efficiency, $L$ is the luminosity and $A_{t \bar{t}}$ the acceptenace rate. \\
To get the trigger effiency just MC was used. It is the quotient between the number of, in the MC as triggerd marked events and the total number of events:
\begin{align}
  \epsilon_{trigger} = \frac{N_{triggerd}}{N_{total}} = 0.86 \pm 0.09
\end{align}
To get the acceptance rate for $t \bar{t}$ events just MC was used. It is the quotient between  the number of selected $t \bar{t}$ events and the total number of simulated $t \bar{t}$ events:
\begin{align}
  A_{t \bar{t}} = \frac{N_{selected}}{N_{total}} = 0.0137 \pm 0.0006
\end{align}
The luminosity is given by the detector:
\begin{align}
  L = 50 \pm 5 \text{ fb$^{-1}$}
\end{align}
Putting everything together one gets the following cross section:
\begin{align}
  \sigma_{t \bar{t}} = 127 \pm 20 \text{ fb}
\end{align}
That is confirms the thoretical prediction:
\begin{align}
  \sigma_{t \bar{t}}^{theory} = 167^{+17}_{-18} \text{ fb}
\end{align}

\section{Z-mass}
To select events with a Z, two isolated muons are required. After selection a Breit-Wigner fit is performed to measure the Z-mass.
\begin{align}
  Z_{mass} = 90.9 \pm 3.7 \text{ GeV}
\end{align}

\begin{figure}[H]
\centering
%% Die dreifachen Klammern {{{ }}} sind nötig, damit Latex auch mit
%% Dateinamen umgehen kann, die Punkte enthalten.
\includegraphics[width=0.75\textwidth]{{{Z_mass_fit}}}
\caption{Breit Wigner fit over the invariant mass of two isolated muons. The peak is near the Z-mass.}
\label{fig:Z_mass_fit}
\end{figure}

In addition a 2D likelihood-scan of the signal strength and the mass of the Z has been performed. The signal strength is equal to the measured cross section devided by the theoretical cross section from the MC simulation. In the MC simulation a $M_Z = 91.188 \text{ GeV}$ had been assumed.
\begin{figure}[H]
\centering
%% Die dreifachen Klammern {{{ }}} sind nötig, damit Latex auch mit
%% Dateinamen umgehen kann, die Punkte enthalten.
\includegraphics[width=0.75\textwidth]{{{contur_plot_Z}}}
\caption{2D likehood-scan of the singal strength vs mass for the Z mass}
\label{fig:Z_mass_con}
\end{figure}


\section{W-mass}
From the selection for top mass events the W mass is calculated by taking the invariant mass of the isolated muon together with the missing transvers energy:
\begin{figure}[H]
\centering
%% Die dreifachen Klammern {{{ }}} sind nötig, damit Latex auch mit
%% Dateinamen umgehen kann, die Punkte enthalten.
\includegraphics[width=0.75\textwidth]{{{W_mass_fit}}}
\caption{Breit Wigner fit over the invariant mass of the isolated muon and the missing transverse energy}
\label{fig:W_mass_fit}
\end{figure}
The Breit-Wigner fit leads to:
\begin{align}
  m_W = 83 \pm 52 \text{ GeV}
\end{align}

\section{Top-mass}
After slection is apllied, the two light jets together with the nearest b-jet give the invaraint mass of the top.

\begin{figure}[H]
\centering
%% Die dreifachen Klammern {{{ }}} sind nötig, damit Latex auch mit
%% Dateinamen umgehen kann, die Punkte enthalten.
\includegraphics[width=0.75\textwidth]{{{t_hadron_mass}}}
\caption{Histogram of the ivariant mass of the two light jets together with the nearest b-jet}
\label{fig:top_had}
\end{figure}
The fit leads to the following mass at one $\sigma$ confidence level:
\begin{align}
  m_{t \bar{t}}^{had} = 172_{-24}^{+60} \text{ GeV}
\end{align}
On the other hand there is the ivaraint mass of the semileptonic decay. The invaraint mass of the muon and the nearest b-jet are added up together with the missing transverse energy:
\begin{figure}[H]
\centering
%% Die dreifachen Klammern {{{ }}} sind nötig, damit Latex auch mit
%% Dateinamen umgehen kann, die Punkte enthalten.
\includegraphics[width=0.75\textwidth]{{{t_lep_m}}}
\caption{Ivariant mass of the muon the nearest b-jet and the missing transverse energy}
\label{fig:top_had}
\end{figure}
The fit leads to the following mass at one $\sigma$ confidence level:
\begin{align}
  m_{t \bar{t}}^{lep} = 184_{-48}^{+48} \text{ GeV}
\end{align}

%\section{Ein paar Tipps \& Tricks}
%Ein nützlicher Tipp fürs schreiben von LaTeX Quelltext, ist es jeden
%  Satz auf einer neuen Zeile zu beginnen.
%Bei langen Zeilen kann der Satz umgebrochen und eingerückt werden.
%Dies macht es einfach einen Satz aus dem PDF auch im Quelltext
%  wiederzufinden.
%Denn meist wirst du das PDF fürs Korrekturlesen verwenden.
%
%\subsection{Die wichtigsten Inhaltselemente}
%Hier folgen nun verschiedene Inhaltselemente.
%Als erstes zeigt \cref{eq:beispiel} den Satz des Pythagoras.
%Wie man Formeln setzt, lernst du wohl am Besten mit Kiles
%  umfangreicher Symboltabelle.
%\begin{equation}
%\label{eq:beispiel} a^2 + b^2 = c^2
%\end{equation}
%
%Im zweiten Absatz dieses Abschnitts folgt nun eine Aufzählung.
%Sie erklärt dir, was du machen musst, damit \cref{fig:beispiel}
%kompiliert:
%\begin{enumerate}[nosep]
%\item Erstelle einen Unterordner \texttt{\color{red} img} (vgl. \texttt{graphicspath})
%\item Lade das Beispielbild\\
%  \url{http://www.physik.uzh.ch/~nchiapol/info1/res/iss+spaceshuttle.jpg}\\
%  herunter und speichere es in diesen Ordner.
%\end{enumerate}
%
%\begin{figure}[H]
% \centering
% %% Die dreifachen Klammern {{{ }}} sind nötig, damit Latex auch mit
% %% Dateinamen umgehen kann, die Punkte enthalten.
% \includegraphics[width=0.45\textwidth]{{{iss+spaceshuttle}}}
% \caption{Das Spaceshuttel und die ISS}
% \label{fig:beispiel}
%\end{figure}
%
%Die Grösse des Bildes ist in unserem Fall in Bruchteilen der
%  Textbreite angegeben.
%Das ist oft die beste Art.
%Sie könnte aber zum Beispiel mit \texttt{[width=8cm]} auch als
%  \SI{8}{\centi\meter} angegeben werden.
%
%Mit den \SI{8}{cm} haben wir auch gleich gezeigt wie man mit \texttt{siunitx}
%  verwendet.
%Das Paket versteht auch Zahlen mit Fehlern wie \SI{60 \pm 5}{kg},
%Wertebereiche wie \SIrange{5}{10}{\degreeCelsius} oder Listen \SIlist{1;2;3;4}{s}.
%
%Schliesslich sollten wir auch noch eine einfache Tabelle
%  zu unserem Beispiel hinzufügen.
%Wir erlauben \cref{tab:beispiel} aber selbst ihre Position zu wählen.
%\begin{table}[tph]
%\centering
%\begin{tabular}{lc}
%\toprule
%Name & Wert\\
%\midrule
%eins & 1\\
%zwei & 2\\
%drei & 3\\
%vier & 4\\
%\bottomrule
%\end{tabular}
%\caption{Unsere Beispiel-Tabelle}
%\label{tab:beispiel}
%\end{table}
%
%\subsection*{Quellenangaben}
%Der \texttt{*} macht dies zu einem unnummerierten Absatz.
%Nun aber zu den versprochenen Quellenangaben.
%Um zu zeigen, wie man in Latex Quellen zitieren kann,
%  sei hier auf ein Hilfsmittel im Web verwiesen.
%Immer wieder nützlich ist das Wiki-Book zu LaTex \cite{wikibook}.
%Es ist sehr umfangreich und enthält viele nützliche Tipps.
%
%\begin{thebibliography}{100}
%\addcontentsline{toc}{section}{Bibliography}
%\bibitem{wikibook} LaTeX, \url{http://en.wikibooks.org/wiki/LaTeX}
%\end{thebibliography}

\end{document}
